\documentclass[a4paper, 12pt]{report}
\usepackage[utf8]{inputenc}
\usepackage[T1]{fontenc}

\usepackage{xcolor}
\usepackage{afterpage}

\usepackage{relsize}
\usepackage{moresize}

\usepackage{graphicx}
\usepackage{geometry}

\setcounter{secnumdepth}{3}
% \setcounter{tocdepth}{3}

% [CHANGE] The title of your thesis. If your thesis has a subtitle, then this
% should appear right below the main title, in a smaller font.
\newcommand{\theTitle}{The first sentence \\
\vspace{0.5em}
the second sentence}
\newcommand{\theSubTitle}{a smaller subtitle}


% [CHANGE] Your full name. In case of multiple names, you can include their
% initials as well, e.g. "Robin G.J. van Achteren".
\newcommand{\theAuthor}{Dewi E. Timman}

% [CHANGE] Your student ID, as this has been assigned to you by the UvA
% administration.
\newcommand{\theStudentID}{12419273}

% [CHANGE] The name of your supervisor(s). Include the titles of your supervisor(s),
% as well as the initials for *all* of his/her first names.
\newcommand{\theSupervisor}{Dr. R. G. Alhama} % Dr. Ing. L. Dorst

% [CHANGE] The address of the institute at which your supervisor is working.
% Be sure to include (1) institute (is appropriate), (2) faculty (if
% appropriate), (3) organisation name, (4) organisation address (2 lines).
\newcommand{\theInstitute}{
Institute for Logic, Language and Computation \\ %Institute for Logic, Language and Computation
Faculty of Science\\
University of Amsterdam\\
Science Park 107 \\ 
1098 XG Amsterdam 
}

% [CHANGE] The semester in which you started your thesis.
\newcommand{\theDate}{\today}

\begin{document}
\pagestyle{empty}
\begin{center}

\vspace{2.5cm}


\begin{Huge}
% see definition at beginning of document
\theTitle
\end{Huge} \\

\vspace{0.5 cm}

\begin{Large}
\theSubTitle
\end{Large}

\vspace{1.5cm}

% see definition at beginning of document
\theAuthor\\
% see definition at beginning of document
\theStudentID

\vspace{1.5cm}

% [DO NOT CHANGE]
Bachelor thesis\\
Credits: 12 EC

\vspace{0.5cm}

% [DO NOT CHANGE] The name of the educational programme.
Bachelor \textit{Cognition, Language and Communication} \\
\vspace{0.25cm}
\includegraphics[width=0.075\paperwidth]{uva_logo} \\
\vspace{0.1cm}

% [DO NOT CHANGE] The address of the educational programme.
University of Amsterdam\\
Faculty of Humanities\\
Spuistraat 134\\
1012 VB Amsterdam

\vspace{2cm}

\emph{Supervisor}\\

% see definition at beginning of document
\theSupervisor

\vspace{0.25cm}

% see definition at beginning of document
\theInstitute

\vspace{1.0cm}

% see definition at beginning of document
\theDate

\end{center}
\newpage

\pagenumbering{arabic}
\setcounter{page}{1}
\pagestyle{plain} 

\chapter*{Abstract}
% % \textbf{Key words:} 
% TODO

% right place?
\tableofcontents


\chapter{Introduction}
% Bring AI and linguistics/cognition together
% How did language originate? How did it become so complex? \\
% Language acquisition

We are able to talk and communicate with each other. 
Our Languages are very complex. 
But how did language originate in the first place, and how do we acquire (new) languages? 

\section{Literature review}
% Important concepts/definitions: agent-based communication, referential communication games, word embedding vectors, LLMs

When researching the emergence of language, \textcolor{orange}{usually \dots}
To do this, it is important to define a few concepts. 
First, about what kind of communication are we talking?
Second, how can we research this communication? 
And Third, in what state do we start the research?

\subsection{Agent-based linguistic communication}
% Communication between two or more agents.
% The agents do not have any prior linguistic knowledge.
% They need to solve a task.
There are different kinds of communication. 
In this research, there will be looked at agent-based linguistics communication. 
This means that there is communication between two or more agents with the use of language. 
\textcolor{orange}{Usually, that means the agents need to solve a task together.} 
To solve the task, agents need to succesfully communicate to each other. 

\subsection{Referential communication games}
% In the context of agent-based communication: agents talk about objects or other entities in a specified world.
% The agents need to come up with a language to commuicate about their world.
To research this agent-based linguistic communication, agents usually participate into a referential communication game. 
\textcolor{orange}{In such a game, agents talk about objects or other entities in a specified world.}
To do this, they need to come up with a language to communicate about they world. 
\textcolor{orange}{Usually, the agents do not have any prior linguistic knowledge.} 

\subsection{Word embedding vectors}
% Represent the world of an agent with embedding vectors from a LLM.
In this research, however, agents do start with prior linguistics knowledge. 
Agents use the word embedding vectors of a large language model (LLM) \textcolor{orange}{in order to \dots}

\subsubsection{LLMs}
% Which LLM is suitable?
The LLM from which the word embedding vectors are chosen, is \textcolor{orange}{\dots. Because \dots}

\section{Current research}
Gap: Research with prior linguistic knowledge -> Do the agents do something with the structure/semantics already present in the word embedding vectors and can this say something about language emergence? \\
RQ: What is the effect of word embeddding vectors from a language model on the emergence of agent-based linguistic communication from referential communication games?


\chapter{Method}
Programming a model and see how it behaves and what makes it behave like that. \\
Dependent variable: output meassured by the model and its evaluation metric -> how often is the communication succesfull? \\
Independent variable: input -> words/sentences \\

\section{Data}
What data? Data preprocessing? Data split?

\subsection{(Data analyse)}
What does the data look like? Is there a data imbalance?

\section{Experimental Design}
What is the setup?

\section{Model description}
What does the model look like? Which equations are used? Diagrams/pseudocode?

\subsection{Hyperparameters}
What are the hyperparameters?

\section{(Optimization)}
What optimizations were done?

\chapter{Results}


\chapter{Conclusion}


\chapter{Discussion}
% Bring AI and linguistics/cognition together

\end{document}